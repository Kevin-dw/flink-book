\documentclass[cn,11pt,chinese]{elegantbook}

\title{Flink实战宝典}
\subtitle{Flink理论与实战}

\author{左元}
\institute{尚硅谷}
\date{April 12, 2020}
\version{0.1}

\extrainfo{完成比完美更重要!}
\logo{logo-blue.png}
\cover{cover.jpg}

\usepackage{array}
\newcommand{\ccr}[1]{\makecell{{\color{#1}\rule{1cm}{1cm}}}}

\begin{document}

\maketitle
\frontmatter

\tableofcontents

\part{Flink的使用}

\chapter{第一章,流式处理理论概述}

\chapter{第二章,Flink框架快速上手}

\chapter{第三章,Flink DataStream API}

\chapter{第四章,基于时间和窗口的算子}

\chapter{第五章,Flink状态编程}

\chapter{第六章,Flink DataSet API}

\chapter{第七章,Flink与外部系统的交互}

\chapter{第八章,Flink Table API \& SQL}

\chapter{第九章,Flink CEP库的使用}

\part{Flink的部署与运维}

\chapter{第十章,Flink应用的监控}

\chapter{第十一章,如何部署Flink集群}

\chapter{第十二章,Flink集群的高可用}

\part{Flink内核与优化}

\chapter{第十三章,Flink运行时架构}

\chapter{第十四章,Flink状态的原理}

\chapter{第十五章,Flink的容错机制}

\chapter{第十六章,Flink作业的调度}

\chapter{第十七章,Flink内存管理的特点}

\chapter{第十八章,Flink的网络IO机制}

\chapter{第十九章,Flink常见优化措施}

\chapter{第二十章,流的去重及其优化}

\part{Flink实时数仓项目}

\end{document}
